%A simple LaTeX syllabus template by Luke Smith.
%I believe this is a variation of a template I got by Mike Hammond...?

%\documentclass[11pt,twocolumn]{article}
\documentclass[11pt,onecolumn]{article}
\usepackage{hyperref}
\usepackage[margin=0.75in]{geometry}
\usepackage{titlesec}
\usepackage{longtable}
\usepackage{gb4e}
%\usepackage[margin=.75in]{geometry}
\usepackage{textcomp}
\pagenumbering{gobble}
%% \usepackage[utf8]{inputenc}
%% \usepackage[english]{babel}
%% \usepackage[usenames, dvipsnames]{color}

%paragraph formatting
\setlength{\parindent}{0pt}
\setlength{\parskip}{3pt}

\usepackage{url}

\hypersetup{
    colorlinks=true,
    linkcolor=blue,
    filecolor=magenta,      
    urlcolor=blue,
}

\titlespacing\subsection{0in}{\parskip}{\parskip}
\titlespacing\section{0in}{\parskip}{\parskip}

%Just fill these in to fill in the basic syllabus information.
\newcommand{\coursename}{Introduction to Bayesian Statistics - STAT 4XX/6XX}
\newcommand{\semester}{Spring 2019}
\newcommand{\roomnumb}{AB 635}
\newcommand{\classtimes}{Mon,Wed 1:00pm - 2:15pm}
\newcommand{\myname}{A.~Grant Schissler}
%\newcommand{\myemail}{\href{mailto:aschissler@unr.edu}{aschissler@unr.edu}}
\newcommand{\myemail}{aschissler@unr.edu}
\newcommand{\office}{DMSC 224}
\newcommand{\officehours}{Tue 2:30pm-3:30pm, Wed 1:30pm-2:30pm, or by appointment}
% \newcommand{\university}{The University of Nevada, Reno}
\renewcommand{\familydefault}{\sfdefault}

\title{\textbf{\coursename} \\Teaching Methods}
% \author{{\university}---{\semester}---{\roomnumb}---{\classtimes}}
%% \author{{\semester}---{\roomnumb}---{\classtimes}}
\date{}

\begin{document}
\maketitle

\vspace{-0.75in}
\noindent\makebox[\linewidth]{\rule{\textwidth}{1pt}}

%% \begin{center}
%% \begin{tabular}{llll}
%% \textbf{Instructor}:&\myname & \textbf{Contact}:&\href{mailto:\myemail}{\myemail}, 775-784-4661 (office)\\
%% \textbf{Office}:&\office & \textbf{Hours}:&\officehours\\
%% \end{tabular}
%% \end{center}

\section{Learning outcome 1}
\textbf{Recall the axioms, basic terms/algebra of probability, including Bayes' Theorem.}

Cognitive level:~Remember

Appropriate methods:~Lecture, interactive lecture, recitation, just-in-time teaching, inquiry based\footnotemark, problem-based learning\footnotemark[\value{footnote}], project-based learning\footnotemark[\value{footnote}], fieldwork

\footnotetext{Depends on the lecture-break tasks, the discussion questions, or the group tasks assigned.}

\section{Learning outcome 2}
\textbf{Model parameters and data using discrete and continuous random variables.}

Cognitive level:~Apply

Appropriate methods:~Interactive lecture\footnotemark, directed discussion\footnotemark[\value{footnote}], writing/speaking exercises, classroom assessment techniques, group work or learning\footnotemark[\value{footnote}], cookbook labs, case method, inquiry based, problem-based, project-based, role plays/simulation, service learning with reflection, fieldwork

\footnotetext{The knowledge acquired may be narrowly focused on the problem or project.}

\section{Learning outcome 3}
\textbf{Conduct Bayesian inference for parameters of discrete and continuous random variables.}

Cognitive level:~Apply and Analyze

Appropriate methods:~Same as above LO 3

\section{Learning outcome 4}
\textbf{Conduct Bayesian inference parameters in linear regression.}

Cognitive level:~Apply and Analyze

Appropriate methods:~Same as above LO 3

\section{Learning outcome 5}
\textbf{Apply computational techniques to conduct Bayesian inference, including Markov Chain Monte Carlo.}

Cognitive level:~Apply and Analyze

Appropriate methods:~Same as above LO 3

\section{Learning outcome 6}
\textbf{Compare Frequentist/Bayesian approaches in statistical inference.}

Cognitive level:~Understand
Appropriate methods:~~Interactive lecture, directed discussion, writing/speaking exercises, classroom assessment techniques, group work or learning, student-peer feedback, cookbook labs, inquiry based, project-based, role plays/simulation

\section{Learning outcome 7, Graduate only}
\textbf{Develop and evaluate a Bayesian model for real world data.}

Cognitive level:~Evaluate and Create

Appropriate methods:~ Writing/speaking exercises, group work, case method, inquiry based, problem-based, project-based, service learning with reflection, fieldwork

\end{document}