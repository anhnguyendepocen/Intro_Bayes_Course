%A simple LaTeX syllabus template by Luke Smith.
%I believe this is a variation of a template I got by Mike Hammond...?

%\documentclass[11pt,twocolumn]{article}
\documentclass[11pt,onecolumn]{article}
\usepackage{hyperref,color}
\usepackage[margin=0.75in]{geometry}
\usepackage{titlesec}
\usepackage{longtable}
\usepackage{gb4e}
%\usepackage[margin=.75in]{geometry}
\usepackage{textcomp}
\pagenumbering{gobble}

%paragraph formatting
\setlength{\parindent}{0pt}
\setlength{\parskip}{6pt}

\usepackage{url}

\hypersetup{
    colorlinks=true,
    linkcolor=blue,
    filecolor=magenta,      
    urlcolor=blue,
}

\titlespacing\subsection{0in}{\parskip}{\parskip}
\titlespacing\section{0in}{\parskip}{\parskip}

%Just fill these in to fill in the basic syllabus information.
\newcommand{\coursename}{Independent Study - STAT 793}
\newcommand{\semester}{Spring 2019}
\newcommand{\roomnumb}{AB 206}
\newcommand{\classtimes}{Mon,Wed 1:00pm - 2:15pm}
\newcommand{\myname}{A.~Grant Schissler}
% \newcommand{\myemail}{\href{mailto:aschissler@unr.edu}{aschissler@unr.edu}}
\newcommand{\myemail}{aschissler@unr.edu}
\newcommand{\office}{DMSC 224}
\newcommand{\officehours}{Tue 3:00pm-4:00pm, Wed 2:30pm-3:30pm, or by appointment}
% \newcommand{\university}{The University of Nevada, Reno}
\renewcommand{\familydefault}{\sfdefault}

\title{\textbf{\coursename}}
% \author{{\university}---{\semester}---{\roomnumb}---{\classtimes}}
\author{{\semester}---{\roomnumb}---{\classtimes}}
\date{}


\begin{document}
\maketitle

\vspace{-0.25in}
\noindent\makebox[\linewidth]{\rule{\textwidth}{1pt}}

\begin{center}
\begin{tabular}{llll}
\textbf{Instructor}:&\myname & \textbf{Contact}:&\href{mailto:\myemail}{\myemail}, 775-784-4661 (office)\\
\textbf{Office}:&\office & \textbf{Hours}:&\officehours\\
\end{tabular}
\end{center}

This 1-hour course supplements a concurrent course introducing Bayesian statistics. Students will complete additional homework exercises and must complete a research project. For this project, students will select an original research paper in Bayesian statistics to read and report on.

\section*{Catalog Description}
Statistical inference using Bayes' Theorem. Topics include posterior analysis for continuous and discrete random variables, prior specification, Bayesian regression, multivariate inference, and posterior sampling through Markov Chain Monte Carlo.

\section*{Course Pre-requisites}
STAT 352 or STAT 467/667 or with instructor approval. STAT 445/645 is a suggested preparation.

\section*{400-level Student Learning Outcomes}
\begin{description}
\item[UG1] Students will be able to demonstrate understanding of the concepts that underlie Bayesian inference and compare the results to frequentist alternatives.
\item[UG2] Students will be able to conduct Bayesian inference analytically and interpret the results.
\item[UG3] Students will be able to perform a Bayesian analysis using professional statistical packages (e.g., \textsf{Minitab}, \textsf{R}, and \textsf{Stan}).
\end{description}

\section*{600-level Student Learning Outcomes}
In addition to the above 400-level outcomes, graduate students will:
\begin{description}
\item[GRAD1] Students will be able to synthesize course concepts to apply Bayesian modeling techniques to real-world data in the pursuit of scientific inquiry.
\end{description}

\section*{Required text}
\emph{Introduction to Bayesian Statistics, 3rd edition}, by William M. Bolstad, James M. Curran\\
Textbook website:\\~\url{https://www.wiley.com/en-us/Introduction+to+Bayesian+Statistics%2C+3rd+Edition-p-9781118091562}

\section*{Assignments}
Exercises will be assigned approximately weekly. You are encouraged to discuss assignments between each other and with instructor. However, the assignment must be completed and submitted individually.

\section*{Midterms}
No additional midterms will be given.

\section*{Final exam}
No additional final exam will be given.

\section*{Exam policy}
None.

\section*{400/600 Students}
As indicated above, the student learning outcomes differ at the 400 and 600 levels. 600-level students must complete a \textbf{term project} in addition to all 400-level requirements.

\section*{Makeup, Late Policy}
Late assignments, exams, and projects will not be graded. Exceptions will be made when a student misses work due to a documented (doctor's note) illness or an extraordinary situation (up to the discretion of the instructor). There will be no early or make-up exams. However, if you need to miss an exam due to participation in a religious holiday or an official university activities (including athletics and other sanctioned activities), you must make arrangements with the instructor at least two weeks prior to the exam in question. Since the late policy is rather strict, \textbf{I'll drop your lowest two grades in the ``Assignments'' category.}

\section*{Grading}

We'll use a point system to evaluate student learning. There are four categories of assignments. Table \ref{tab:points} shows the total points possible within each category before dropping any low scores.

\begin{table}[h]
  \begin{center}
  \begin{tabular}{l|c}	%For grade items (quizzes, homework, etc.)
Item& Total points \\\hline\hline
  Assignments& 60 \\
  Term project & 50 \\
  \end{tabular}
  \caption{Total points available within each assignment category. There will be 12 Assignments worth 5 points each. \label{tab:points}}
\end{center}

\end{table}

The sum of all points earned determine the final letter grades, according to the thresholds listed in Table \ref{tab:grade}.

\begin{table}[h]
\begin{center}
  \begin{tabular}{l|l}
    Letter grade & Threshold\\\hline\hline
A ($\geq 90\%$)&90 or above\\
B ($\geq 80\%$)&80 to 89\\
C ($\geq 70\%$)&70 to 79\\
D ($\geq 60\%$)&60 to 69\\
F ($< 60\%$)&59 or below
  \end{tabular}
  \caption{Conversion table between points and letter grades, after dropping the two lowest ``Assignments'' grades .\label{tab:grade}}
\end{center}

\end{table}

The instructor reserves the right to deviate from the above thresholds, including assigning $+$ or $-$.

\section*{Diversity Statement}
The University of Nevada, Reno is committed to providing a safe learning and work environment for all. If you believe you have experienced discrimination, sexual harassment, sexual assault, domestic/dating violence, or stalking, whether on or off campus, or need information related to immigration concerns, please contact the University’s Equal Opportunity \& Title IX Office at (775) 784-1547. Resources and interim measures are available to assist you. For more information, please visit \url{http:www.unr.edu/equal-opportunity-title-ix}.

\section*{Disability Statement}
Any student with a disability needing academic adjustments or accommodations is requested to speak with the \href{http:www.unr.edu/drc}{Disability Resource Center} as soon as possible to arrange for appropriate accommodations.

\section*{Academic Conduct}
No laptops, cell phones, mp3 players, or other electronics are to be used for personal reasons in class. If you are being disruptive during class you will be asked to leave. Disruptions in this context include inadequate participation. You must come to class on time and stay until the end of lecture. Tardy students will not be admitted to class. Please visit \url{http:www.unr.edu/student-conduct} for our official student code of conduct.

\section*{Academic Success Services}
Your student fees cover usage of the University Math Center, University Tutoring Center, and University Writing Center. These centers support your classroom learning; it is your responsibility to take advantage of their services. Keep in mind that seeking help outside of class is the sign of a responsible and successful student

\section*{University Recording Policy}
Surreptitious or covert videotaping of class or unauthorized audio recording of class is prohibited by law and by Board of Regents policy. This class may be videotaped or audio recorded only with the written permission of the instructor. In order to accommodate students with disabilities, some students may have been given permission to record class lectures and discussions. Therefore, students should understand that their comments during class may be recorded.

\section*{Academic Dishonesty}
Cheating, plagiarism, or otherwise obtaining grades under false pretenses constitutes academic dishonesty according to the code of this university. Academic dishonesty will not be tolerated and penalties can include canceling a student’s enrollment without a grade or giving an F for the assignment or for the entire course. See the University Academic Standards policy: \href{https://www.unr.edu/administrative-manual/6000-6999-curricula-teaching-research/instruction-research-procedures/6502-academic-standards}{UAM 6,502}.

\section*{Tentative course schedule}
\begin{center}
  \begin{tabular}{|c|c|c|c|}
    \hline
    Week & Monday & Wednesday & Notes \\
    \hline
    1 & MLK Day & Why Bayes?/Syllabus/expectations & Ch.1-3\\
    \hline
    2 & Probability & Probability & Ch.4 \\
    \hline
    3 & Discrete random variables & Discrete RVs & Ch.5,6 \\ \\
    \hline
    4 & Discrete RVs& \textbf{Midterm 1} & Ch.5,6 \\ \\
    \hline
    5 & Continuous RVs & Continuous RVs & Ch.7-10 \\
    \hline
    6 & Review session & President's Day & \\
    \hline
    7 & Continuous RVs & Continuous RVs & Ch.7-10 \\
    \hline
    8 & Continuous RVs & \textbf{Midterm 2} & Ch.7-10 \\
    \hline
    9 & -- & -- & Spring break  \\
    \hline
    10 & Continuous RVs II & Continuous RVs II & Ch.11,12,13 \\
    \hline
    11 & Continuous RVs II & Continuous RVs II & Ch.11,12,13 \\
    \hline
    12 & Continuous RVs II & \textbf{Midterm 3} & \\
    \hline
    13 & Simple linear regression & Simple linear regression & Ch.14 \\
    \hline
    14 & Computing/MCMC & Computing/MCMC & Ch.20\\
    \hline
    15 & Computing/MCMC & \textbf{Midterm 4} & Ch.20\\
    \hline
    16 & Student talks & Prep Day &  \\
    \hline
    17 & \textbf{Take-home final exam} & & \\
    \hline
\end{tabular}
\end{center}

\end{document}