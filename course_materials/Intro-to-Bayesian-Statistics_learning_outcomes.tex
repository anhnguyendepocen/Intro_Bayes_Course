%A simple LaTeX syllabus template by Luke Smith.
%I believe this is a variation of a template I got by Mike Hammond...?

%\documentclass[11pt,twocolumn]{article}
\documentclass[11pt,onecolumn]{article}
\usepackage{hyperref}
\usepackage[margin=0.75in]{geometry}
\usepackage{titlesec}
\usepackage{longtable}
\usepackage{gb4e}
%\usepackage[margin=.75in]{geometry}
\usepackage{textcomp}
\pagenumbering{gobble}
%% \usepackage[utf8]{inputenc}
%% \usepackage[english]{babel}
%% \usepackage[usenames, dvipsnames]{color}

%paragraph formatting
\setlength{\parindent}{0pt}
\setlength{\parskip}{3pt}

\usepackage{url}

\hypersetup{
    colorlinks=true,
    linkcolor=blue,
    filecolor=magenta,      
    urlcolor=blue,
}

\titlespacing\subsection{0in}{\parskip}{\parskip}
\titlespacing\section{0in}{\parskip}{\parskip}

%Just fill these in to fill in the basic syllabus information.
\newcommand{\coursename}{Introduction to Bayesian Statistics - STAT 4XX/6XX}
\newcommand{\semester}{Spring 2019}
\newcommand{\roomnumb}{AB 635}
\newcommand{\classtimes}{Mon,Wed 1:00pm - 2:15pm}
\newcommand{\myname}{A.~Grant Schissler}
%\newcommand{\myemail}{\href{mailto:aschissler@unr.edu}{aschissler@unr.edu}}
\newcommand{\myemail}{aschissler@unr.edu}
\newcommand{\office}{DMSC 224}
\newcommand{\officehours}{Tue 2:30pm-3:30pm, Wed 1:30pm-2:30pm, or by appointment}
% \newcommand{\university}{The University of Nevada, Reno}
\renewcommand{\familydefault}{\sfdefault}

\title{\textbf{\coursename}}
% \author{{\university}---{\semester}---{\roomnumb}---{\classtimes}}
\author{{\semester}---{\roomnumb}---{\classtimes}}
\date{}


\begin{document}
\maketitle

\vspace{-0.25in}
\noindent\makebox[\linewidth]{\rule{\textwidth}{1pt}}

\begin{center}
\begin{tabular}{llll}
\textbf{Instructor}:&\myname & \textbf{Contact}:&\href{mailto:\myemail}{\myemail}, 775-784-4661 (office)\\
\textbf{Office}:&\office & \textbf{Hours}:&\officehours\\
\end{tabular}
\end{center}

\section*{Catalog Description}
Statistical inference using Bayes’ Theorem. Topics include Bayesian/frequentist comparison, posterior analysis for continuous and discrete random variables, prior specification, Bayesian regression, multivariate inference, and posterior sampling through Markov Chain Monte Carlo.

\section*{Broad student learning outcomes}

\begin{description}
\item[UG1] Students will be able to demonstrate understanding of the concepts that underlie Bayesian inference and compare the results to frequentist alternatives.
\item[UG2] Students will be able to conduct Bayesian inference analytically and interpret the results.
\item[UG3] Students will be able to perform a Bayesian analysis using professional statistical packages (e.g., \textsf{Minitab}, \textsf{R}, and \textsf{Stan}).
\item[GRAD1] Students will be able to synthesize course concepts to apply Bayesian modeling techniques to real-world data in the pursuit of scientific inquiry.
\end{description}

\section*{Learning outcomes}

Students will be able to $\ldots$

\begin{enumerate}
  \itemsep0em

  %% The changes below reflect an attempt to get to 5 - 7 core SLOs.
%% \item explain the role of statistics in science using sampling theory.
%% \item describe and summarize data numerically and visually.
%% \item apply probability theory to analyze uncertainty in real world problems.
%% \item model parameters and data using discrete random variables.
%% \item conduct Bayesian inference for parameters of discrete random variables.
%% \item model parameters and data using continuous random variables.
%% \item conduct Bayesian inference for parameters of continuous random variables.
  \item recall the axioms, basic terms/algebra of probability, including Bayes' Theorem.
\item model parameters and data using discrete and continuous random variables.
\item conduct Bayesian inference for parameters of discrete and continuous random variables.
%% \item conduct Bayesian inference for a binomial proportion.
%% \item compare Frequentist/Bayes approaches for inferring a binomial proportion.
%% \item conduct Bayesian inference for a Normal mean parameter.
%% \item compare Frequentist/Bayes approaches for inferring a Normal mean parameter.
\item conduct Bayesian inference parameters in linear regression.
%% \item conduct Bayesian inference for a Normal variance parameter.
%% \item will apply robust Bayes methods for prior misspecification.
%% \item conduct Bayesian inference for parameters in a multivariate Normal.
\item apply computational techniques to conduct Bayesian inference, including Markov Chain Monte Carlo.
\item compare Frequentist/Bayesian approaches in statistical inference.
\item develop and evaluate a Bayesian model for real world data.

\end{enumerate}


\end{document}