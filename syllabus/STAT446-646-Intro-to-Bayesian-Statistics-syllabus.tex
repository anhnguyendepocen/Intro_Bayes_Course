%A simple LaTeX syllabus template by Luke Smith.
%I believe this is a variation of a template I got by Mike Hammond...?

%\documentclass[11pt,twocolumn]{article}
\documentclass[11pt,onecolumn]{article}
\usepackage{hyperref,color}
\usepackage[margin=0.75in]{geometry}
\usepackage{titlesec}
\usepackage{longtable}
\usepackage{gb4e}
%\usepackage[margin=.75in]{geometry}
\usepackage{textcomp}
\pagenumbering{gobble}

%paragraph formatting
\setlength{\parindent}{0pt}
\setlength{\parskip}{6pt}

\usepackage{url}

\hypersetup{
    colorlinks=true,
    linkcolor=blue,
    filecolor=magenta,      
    urlcolor=blue,
}

\titlespacing\subsection{0in}{\parskip}{\parskip}
\titlespacing\section{0in}{\parskip}{\parskip}

%Just fill these in to fill in the basic syllabus information.
\newcommand{\coursename}{Introduction to Bayesian Statistics - STAT 446/646}
\newcommand{\semester}{Spring 2019}
\newcommand{\roomnumb}{AB 206}
\newcommand{\classtimes}{Mon,Wed 1:00pm - 2:15pm}
\newcommand{\myname}{A.~Grant Schissler}
% \newcommand{\myemail}{\href{mailto:aschissler@unr.edu}{aschissler@unr.edu}}
\newcommand{\myemail}{aschissler@unr.edu}
\newcommand{\office}{DMSC 224}
\newcommand{\officehours}{Mon 2:30pm-3:30pm, Wed 3:30pm-4:30pm, or by appointment}
% \newcommand{\university}{The University of Nevada, Reno}
\renewcommand{\familydefault}{\sfdefault}

\title{\textbf{\coursename}}
% \author{{\university}---{\semester}---{\roomnumb}---{\classtimes}}
\author{{\semester}---{\roomnumb}---{\classtimes}}
\date{}


\begin{document}
\maketitle

\vspace{-0.25in}
\noindent\makebox[\linewidth]{\rule{\textwidth}{1pt}}

\begin{center}
\begin{tabular}{llll}
\textbf{Instructor}:&\myname & \textbf{Contact}:&\href{mailto:\myemail}{\myemail}, 775-784-4661 (office)\\
\textbf{Office}:&\office & \textbf{Hours}:&\officehours\\
\end{tabular}
\end{center}

This course aims to introduce Bayesian statistics to students at an early stage, provided they have a reasonable mathematical background. As such, the range of topics is similar to an introductory statistics course for a technical audience with an emphasis on statistical inference. Along those lines, the course will describe Bayesian methods and show how they often compare favorable to frequentist alternatives. Topics include Bayesian inference for discrete and continuous univariate random variables, multivariate normal, and linear regression. We will implement methods analytically and using software such as Minitab, R, and Stan.

The latter portions of the course focus on implementing more advanced Bayesian methods through modern computational techniques to sample posterior distributions including Markov Chain Monte Carlo (MCMC) and Hamiltonian Monte Carlo (HMC). The aim will be to provide the concepts and technical skills to conduct a realistic Bayesian data analysis. This will serve advanced undergraduate and graduate students in their research projects. As such, the later topics may be challenging to those without experience in computer programming (although it is not formally required for the course).

\section*{Catalog Description}
Statistical inference using Bayes' Theorem. Topics include posterior analysis for continuous and discrete random variables, prior specification, Bayesian regression, multivariate inference, and posterior sampling through Markov Chain Monte Carlo.

\section*{Course Pre-requisites}
STAT 352 or STAT 467/667 or with instructor approval. STAT 445/645 is a suggested preparation.

\section*{400-level Student Learning Outcomes}
\begin{description}
\item[UG1] Students will be able to demonstrate understanding of the concepts that underlie Bayesian inference and compare the results to frequentist alternatives.
\item[UG2] Students will be able to conduct Bayesian inference analytically and interpret the results.
\item[UG3] Students will be able to perform a Bayesian analysis using professional statistical packages (e.g., \textsf{Minitab}, \textsf{R}, and \textsf{Stan}).
\end{description}

\section*{600-level Student Learning Outcomes}
In addition to the above 400-level outcomes, graduate students will:
\begin{description}
\item[GRAD1] Students will be able to synthesize course concepts to apply Bayesian modeling techniques to real-world data in the pursuit of scientific inquiry.
\end{description}

\section*{Required text}
\emph{Introduction to Bayesian Statistics, 3rd edition}, by William M. Bolstad, James M. Curran\\
Textbook website:\\~\url{https://www.wiley.com/en-us/Introduction+to+Bayesian+Statistics%2C+3rd+Edition-p-9781118091562}

\section*{Assignments}
Exercises will be assigned approximately weekly. You are encouraged to discuss assignments between each other and with instructor. However, the works must be completed and submitted individually.

\section*{Exam policy} You will be allowed at one 8.5x11in page of handwritten (on both sides) notes for each midterm and three such pages for the final exam. If you believe that your grade for exam or assignment is incorrect, contact instructor at the office hours with a rational justification. All such requests must be submitted to instructor within one week after a grade is announced; late requests will not be granted. The final decision about new grade is made by the instructor. Please understand that everyone can make a mistake, and that mistakes can go both ways: higher or lower than original grade.

\section*{Midterms}
There will be two midterms, the first on Thursday, March 1, and the second on Thursday, April 12.

\section*{Final exam}
A comprehensive final examination will be held on May 15, 9:00am - 11:00am, AB 206.

\section*{400/600 Students}
As indicated above, the student learning outcomes differ at the 400 and 600 levels. Assignments and exams for students enrolled at the 400 level and 600 level will also differ. Specifically, graduate students will be expected to complete more exercises and exercises of greater difficulty on assignments. Exams scoring will demand a higher standard of correctness and communication. Students enrolled at the 600 level will also be required to complete a \textbf{term project} due May 14 in class.

\section*{Makeup, Late Policy}
Late assignments, exams, and projects will not be graded. Exceptions will be made when a student misses work due to documented (doctor's note) illness or other extraordinary situation, up to the discretion of the instructor. There will be no early or make-up exams. However, if you need to miss an exam due to participation in official university activities (including athletics and other sanctioned activities), you must make arrangements with the instructor at least two weeks prior to the exam in question. Since the late policy is rather strict, I will drop your lowest two grades in the ``Assignments'' category as a safety factor for emergencies.

\clearpage
\section*{Grading}
The final grades will be determined using the following percentages:

\begin{center}
\begin{tabular}{cc}
\begin{tabular}{l|c|r}	%For grade items (quizzes, homework, etc.)
Item&400-level& 600-level\\\hline\hline
  Assignments&40\% & 40\%\\
  Midterm Exams&40\%& 20\%\\
  Final Exam&20\%& 20\%\\
  Term project &-- &20\%\\
\end{tabular}
&
\begin{tabular}{ll}
A&90-100\\
B&80-89\\
C&70-79\\
D&60-69\\
F&59 or below
\end{tabular}
\end{tabular}
\end{center}

The instructor reserves the right to deviate from the above percentages in special cases, including assigning borderline cases (generally this could be +/- 3\% points) may be given a $+$ or $-$ within the above intervals or by increasing the letter grade.

\section*{Diversity Statement}
The University of Nevada, Reno is committed to providing a safe learning and work environment for all. If you believe you have experienced discrimination, sexual harassment, sexual assault, domestic/dating violence, or stalking, whether on or off campus, or need information related to immigration concerns, please contact the University’s Equal Opportunity \& Title IX Office at (775) 784-1547. Resources and interim measures are available to assist you. For more information, please visit \url{http:www.unr.edu/equal-opportunity-title-ix}.

\section*{Disability Statement}
Any student with a disability needing academic adjustments or accommodations is requested to speak with the \href{http:www.unr.edu/drc}{Disability Resource Center} as soon as possible to arrange for appropriate accommodations.

\section*{Academic Conduct}
No laptops, cell phones, mp3 players, or other electronics are to be used for personal reasons in class. If you are being disruptive during class you will be asked to leave. Disruptions in this context include inadequate participation. You must come to class on time and stay until the end of lecture. Tardy students will not be admitted to class. Please visit \url{http:www.unr.edu/student-conduct} for our official student code of conduct.

\section*{Academic Success Services}
Your student fees cover usage of the University Math Center, University Tutoring Center, and University Writing Center. These centers support your classroom learning; it is your responsibility to take advantage of their services. Keep in mind that seeking help outside of class is the sign of a responsible and successful student

\newpage
\section*{University Recording Policy}
Surreptitious or covert videotaping of class or unauthorized audio recording of class is prohibited by law and by Board of Regents policy. This class may be videotaped or audio recorded only with the written permission of the instructor. In order to accommodate students with disabilities, some students may have been given permission to record class lectures and discussions. Therefore, students should understand that their comments during class may be recorded.

\section*{Academic Dishonesty}
Cheating, plagiarism, or otherwise obtaining grades under false pretenses constitutes academic dishonesty according to the code of this university. Academic dishonesty will not be tolerated and penalties can include canceling a student’s enrollment without a grade or giving an F for the assignment or for the entire course. See the University Academic Standards policy: \href{https://www.unr.edu/administrative-manual/6000-6999-curricula-teaching-research/instruction-research-procedures/6502-academic-standards}{UAM 6,502}.

\section*{Tentative course schedule}
\begin{center}
  \begin{tabular}{|c|c|c|c|}
    \hline
    Week & Topic1 & Topic2 & Notes \\
    \hline
    1 & Syllabus/expectations & Statistical science review & Ch.1,2 \\
    \hline
    2 & Summarizing data & Logic, probability, uncertainty & Ch.3,4 \\
    \hline
    3 & Discrete random variables & Inference for discrete random variables & Ch.5,6 \\
    \hline
    4 & Continuous random variables & Inference for binomial proportion & Ch.7,8 \\
    \hline
    5 & Frequentist/Bayes comparison for proportion & Inference for Poisson & Ch.9,10 \\
    \hline
    6 & Review session & \textbf{Midterm 1} & \\
    \hline
    7 & Inference for Normal mean & Frequentist/Bayes comparison for mean & Ch.11,12 \\
    \hline
    8 & Inference for difference in means & con'd & Ch.13 \\
    \hline
    9 & -- & -- & Spring break  \\
    \hline
    10 & Bayesian Simple Linear Regression & con'd & Ch.14 \\
    \hline
    11 & Inference for Standard Deviation & Robust methods & Ch.15,16 \\
    \hline
    12 & Review session & \textbf{Midterm 2} & \\
    \hline
    13 & Inference Normal unknown mean and variance & con'd & Ch.17 \\
    \hline
    14 & Inference Multivariate Normal & con'd & Ch.18\\
    \hline
    15 & Bayesian Multiple Linear Regression & con'd & Ch.19\\
    \hline
    16 & Bayes Computation/MCMC & HMC/Stan & Ch.20 \\
    \hline
    17 & \textbf{Final exam} & & \\
    \hline
\end{tabular}
\end{center}

\end{document}