%A simple LaTeX syllabus template by Luke Smith.
%I believe this is a variation of a template I got by Mike Hammond...?

%\documentclass[11pt,twocolumn]{article}
\documentclass[11pt,onecolumn]{article}
\usepackage{hyperref,color}
\usepackage[margin=0.75in]{geometry}
\usepackage{titlesec}
\usepackage{longtable}
\usepackage{gb4e}
%\usepackage[margin=.75in]{geometry}
\usepackage{textcomp}
\pagenumbering{gobble}

%paragraph formatting
\setlength{\parindent}{0pt}
\setlength{\parskip}{7pt}

\hypersetup{
    colorlinks=true,
    linkcolor=blue,
    filecolor=magenta,      
    urlcolor=blue,
}

\titlespacing\subsection{0in}{\parskip}{\parskip}
\titlespacing\section{0in}{\parskip}{\parskip}

%Just fill these in to fill in the basic syllabus information.
\newcommand{\coursename}{Introduction to Bayesian Analysis - STAT 446/646}
\newcommand{\semester}{Spring 2019}
\newcommand{\roomnumb}{DMSC 106}
\newcommand{\classtimes}{Tue,Thu 3:00pm - 4:15pm}
\newcommand{\myname}{A.~Grant Schissler}
% \newcommand{\myemail}{\href{mailto:aschissler@unr.edu}{aschissler@unr.edu}}
\newcommand{\myemail}{aschissler@unr.edu}
\newcommand{\office}{DMSC 224}
\newcommand{\officehours}{Tuesdays, Wednesdays 1pm-3pm, or by appointment}
% \newcommand{\university}{The University of Nevada, Reno}
\renewcommand{\familydefault}{\sfdefault}

\title{\textbf{\coursename}}
% \author{{\university}---{\semester}---{\roomnumb}---{\classtimes}}
\author{{\semester}---{\roomnumb}---{\classtimes}}
\date{}


\begin{document}
\maketitle

\vspace{-0.25in}
\noindent\makebox[\linewidth]{\rule{\textwidth}{1pt}}

\begin{center}
\begin{tabular}{llll}
\textbf{Instructor}:&\myname & \textbf{Contact}:&\href{mailto:\myemail}{\myemail}, 775-784-4661 (office)\\
\textbf{Office}:&\office & \textbf{Hours}:&\officehours\\
\end{tabular}
\end{center}

This course will provide students with the concepts and technical skills to conduct a modern Bayesian data analysis. Such an analysis offers a flexible and powerful approach to modeling data and scientific inquiry. The target audience has had some linear algebra, calculus, and computer programming along one or two statistics classes. Throughout, an emphasis will be placed modeling data using probability distributions, Monte Carlo simulation, and model evaluation. 

The approach to instruction is largely computational in both the readings and assigned class activities. Further mathematical depth will be discussed but not emphasized. Information theory and entropy will be introduced as motivation for model construction and evaluation. Further topics include the basics of Bayesian analysis, Markov Chain Monte Carlo (MCMC), regression through multilevel (hierarchical) models, measurement error, missing data, and Gaussian process models for spatial and network autocorrelation. The main goal is to empower learners to confidently perform and communicate a challenging Bayesian analysis using state-of-the-art statistical computing and modeling techniques.

\section*{Catalog Description}
Bayesian data analysis; posterior approximation; Markov Chain Monte Carlo analysis and remediation; prior specification; Hamiltonian Monte Carlo; information theory and criterion; maximum entropy; regularization; generalized multilevel (hierarchical) regression; mixture modeling.

\section*{Course Pre-requisites}
STAT 352 or STAT 467 or STAT 667 or with instructor approval. Enrollment in STAT 445/645 is a suggested preparation.

\section*{400-level Student Learning Outcomes}
\begin{description}
\item[UG1] Students will be able to demonstrate understanding of the concepts that underly modern methods of Bayesian analysis, and critically assess the assumptions associated with different statistical models.
\item[UG2] Students will be able to interpret and discuss the results of Bayesian analyses.
\item[UG3] Students will be able to perform a modern Bayesian analysis using professional statistical packages (e.g., \textsf{R} and \textsf{Stan}).
\end{description}

\section*{600-level Student Learning Outcomes}
\begin{description}
\item[GRAD1] Students will be able to synthesize course concepts to apply Bayesian modeling techniques to real-world data to facilitate scientific inquiry.
\end{description}

\section*{Required text}
\emph{Statistical Rethinking}, by Richard McElreath. \\
Textbook website:~\url{http://xcelab.net/rm/statistical-rethinking/}

\section*{Assignments}
Exercises will be assigned approximately weekly. You are encouraged to discuss  assignments between each other and with instructor. However, the works must be completed and submitted individually.

\section*{Exam policy} You will be allowed at one 8.5x11in page of handwritten (on both sides) notes for each midterm and three such pages for the final exam. If you believe that your grade for exam or assignment is incorrect, contact instructor at the office hours with a rational justification. All such requests must be submitted to instructor within one week after a grade is announced; late requests will not be granted. The final decision about new grade is made by the instructor. Please understand that everyone can make a mistake, and that mistakes can go both ways: higher or lower than original grade.

\section*{Midterms}
There will be two midterms, the first on Thursday, March 1, and the second on Thursday, April 12.

\section*{Final exam}
A comprehensive final examination will be held on May 15, 9:00am - 11:00am, DMS 106.

\section*{400/600 Students}
As indicated above, the student learning outcomes differ at the 400 and 600 levels. Assignments and exams for students enrolled at the 400 level and 600 level will also differ. Specifically, graduate students will be expected to complete more exercises and exercises of greater difficulty on assignments. Exams will be of higher difficulty. Students enrolled at the 600 level will also be required to complete a \textbf{term project} due May 14 in class.

\section*{Makeup, Late Policy}
Late assignments, exams, and projects will not be graded. Exceptions will be made when a student misses work due to documented (doctor's note) illness or other extraordinary situation, up to the discretion of the instructor. There will be no early or make-up exams. However, if you need to miss an exam due to participation in official university activities (including athletics and other sanctioned activities), you must make arrangements with the instructor at least two weeks prior to the exam in question. Since the late policy is rather strict, I will drop your lowest two grades in the ``Assignments'' category as a safety factor for emergencies.

\section*{Grading}
The final grades will be determined using the following percentages:

\begin{center}
\begin{tabular}{cc}
\begin{tabular}{l|c|r}	%For grade items (quizzes, homework, etc.)
Item&400-level& 600-level\\\hline\hline
  Assignments&60\% & 40\%\\
  Midterm Exams&20\%& 20\%\\
  Final Exam&20\%& 20\%\\
  Term project &-- &20\%\\
\end{tabular}
&
\begin{tabular}{ll}
A&90-100\\
B&80-89\\
C&70-79\\
D&60-69\\
F&59 or below
\end{tabular}
\end{tabular}
\end{center}

The instructor reserves the right to deviate from the above percentages in special cases, including borderline cases (generally this could be +/- 3\% points) may be given a + or − within the above intervals or increasing the letter grade.

\section*{Diversity Statement}
The University of Nevada, Reno is committed to providing a safe learning and work environment for all. If you believe you have experienced discrimination, sexual harassment, sexual assault, domestic/dating violence, or stalking, whether on or off campus, or need information related to immigration concerns, please contact the University’s Equal Opportunity \& Title IX Office at (775) 784-1547. Resources and interim measures are available to assist you. For more information, please visit \url{http:www.unr.edu/equal-opportunity-title-ix}.

\section*{Disability Statement}
The Department of Mathematics and Statistics supports providing equal access for students with disabilities. Any student with a disability needing academic adjustments or accommodations is requested to speak with me or the Disability Resource Center (PSAC 230, \url{http:www.unr.edu/drc}) as soon as possible to arrange for appropriate accommodations.

\section*{Academic Conduct}
No laptops, cell phones, mp3 players, or other electronics are to be used for personal reasons in class. If you are being disruptive during class you will be asked to leave. Disruptions in this context include inadequate participation. You must come to class on time and stay until the end of lecture. Tardy students will not be admitted to class. Please visit \url{http:www.unr.edu/student-conduct} for our official student code of conduct.

\section*{Academic Success Services}
A common habit among successful students is to seek help outside of the classroom. Your student fees cover use of the Math Center (784-4433 or \url{http:www.unr.edu/mathcenter}), Tutoring Center (784-6801 or \url{http:www.unr.edu/tutoring-center}), and University Writing Center (784-6030 or \url{http:www.unr.edu/writing-center}). These centers support your classroom learning; it is your responsibility to take advantage of their services.

\newpage
\section*{University Recording Policy}
Surreptitious or covert videotaping of class or unauthorized audio recording of class is prohibited by law and by Board of Regents policy. This class may be videotaped or audio recorded only with the written permission of the instructor. In order to accommodate students with disabilities, some students may have been given permission to record class lectures and discussions. Therefore, students should understand that their comments during class may be recorded.

\section*{Academic Dishonesty}
Cheating, plagiarism, or otherwise obtaining grades under false pretenses constitutes academic dishonesty according to the code of this university. Academic dishonesty will not be tolerated and penalties can include canceling a student’s enrollment without a grade or giving an F for the assignment or for the entire course. For more details, see the University of Nevada, Reno general catalog.


\section*{Tentative course schedule}
\begin{center}
  \begin{tabular}{|c|c|c|}
    \hline
    Week & Topic & Notes \\
    \hline
    1 & Syllabus and basics of Bayesian analysis & \\
    \hline
    2 & Bayesian computation, sampling the posterior & \\
    \hline
    3 & Linear Models & \\
    \hline
    4 & Multivariate Linear Models & \\
    \hline
    5 & Overfitting and Model Comparison & \\
    \hline
    6 & Review and \textbf{Midterm 1} & \\
    \hline
    7 & Interactions in models & \\
    \hline
    8 & Markov Chain Monte Carlo Estimation and Hamiltonian Monte Carlo & \\
    \hline
    9 & -- & Spring break\\
    \hline
    10 & Entropy and the Generalized Linear Model & \\
    \hline
    11 & Counting and Classification & \\
    \hline
    12 & Review and \textbf{Midterm 2} & \\
    \hline
    13 & Mixtures distributions & \\
    \hline
    14 & Generalized Multilevel (Hierarchical) Models & \\
    \hline
    15 & Multivariate analysis (Covariance/Correlation) & \\
    \hline
    16 & Misc. (e.g. Missing Data and Measurement error) & \\
    \hline
    17 & \textbf{Final exam} & \\
    \hline
\end{tabular}
\end{center}

\end{document}